\documentclass[11pt]{article}
\usepackage[utf8]{inputenc}
\usepackage[vietnamese]{babel}
\usepackage[margin=1in]{geometry}
\usepackage{graphicx}
\usepackage{booktabs}
\usepackage{listings}
\usepackage{xcolor}
\usepackage{hyperref}
\hypersetup{
    colorlinks=true,
    linkcolor=black,
    urlcolor=blue
}

\lstset{
    basicstyle=\ttfamily\small,
    breaklines=true,
    frame=single,
    backgroundcolor=\color{gray!10}
}

\title{Báo Cáo Xây Dựng Hệ Thống Kiểm Chứng Video\\cho Bộ Lọc Sobel Trên FPGA}
\author{Dự Án Sobel}
\date{\today}

\begin{document}

\maketitle

\tableofcontents
\newpage

\section{Giới Thiệu}

\subsection{Mục Tiêu Đề Tài}
Xây dựng một luồng kiểm chứng hoàn chỉnh và tự động cho bộ xử lý phát hiện biên Sobel được thiết kế bằng Verilog RTL, nhằm đảm bảo chức năng đúng trước khi tổng hợp lên FPGA. Hệ thống kiểm chứng sử dụng video thực tế làm đầu vào, mô phỏng RTL, và so sánh kết quả với mô hình phần mềm chuẩn.

\subsection{Động Lực}
Các phương pháp kiểm tra truyền thống (testbench ngẫu nhiên, golden vectors tĩnh) không phản ánh đầy đủ hành vi của hệ thống khi xử lý chuỗi video liên tục. Luồng kiểm chứng dựa trên video thực cho phép:
\begin{itemize}
    \item Đánh giá khả năng xử lý pipeline liên tục qua nhiều khung hình
    \item Phát hiện lỗi liên quan đến đồng bộ hóa dòng dữ liệu (vsync, href)
    \item Cung cấp trực quan hóa trực tiếp (video diff) để phân tích nhanh
    \item Tích hợp dễ dàng vào quy trình CI/CD cho regression testing
\end{itemize}

\section{Kiến Trúc Hệ Thống}

\subsection{Tổng Quan Luồng Xử Lý}
Hệ thống kiểm chứng bao gồm 4 giai đoạn chính (xem Hình~\ref{fig:architecture}):

\begin{enumerate}
    \item \textbf{Chuẩn bị dữ liệu kích thích:} Script Python \texttt{prep\_video\_rgb565.py} đọc video nguồn, chuyển đổi sang định dạng RGB565 (16-bit), và xuất ra file nhị phân kèm metadata để testbench Verilog sử dụng.
    
    \item \textbf{Mô phỏng RTL:} Testbench \texttt{tb\_sobel\_video.v} phát lại stream RGB565, điều khiển module \texttt{sobel\_processor}, và ghi kết quả đầu ra vào file binary.
    
    \item \textbf{Mô hình tham chiếu phần mềm:} Script \texttt{compare\_sobel\_output.py} xây dựng lại chuỗi xám từ RGB565 đầu vào, áp dụng bộ lọc Sobel với các tham số giống hệt RTL, và so sánh từng pixel với kết quả mô phỏng.
    
    \item \textbf{Tự động hóa:} Target \texttt{make video} trong Makefile điều phối toàn bộ các bước, sinh ra báo cáo JSON và video trực quan hóa sự khác biệt.
\end{enumerate}

\begin{figure}[htbp]
    \centering
    \includegraphics[width=0.95\textwidth]{sobel_architecture_diagram.pdf}
    \caption{Sơ đồ kiến trúc tổng thể của hệ thống kiểm chứng video Sobel. Luồng dữ liệu bắt đầu từ video nguồn, qua các script Python chuẩn bị dữ liệu, mô phỏng RTL với testbench Verilog, so sánh với mô hình phần mềm, và cuối cùng sinh ra báo cáo metrics cùng video diff.}
    \label{fig:architecture}
\end{figure}

\subsection{Các Module RTL Chính}
\begin{itemize}
    \item \texttt{sobel\_processor.v}: Module top-level tích hợp toàn bộ pipeline
    \item \texttt{rgb\_to\_gray.v}: Chuyển đổi RGB565 sang grayscale 8-bit (weighted average: $0.299R + 0.587G + 0.114B$)
    \item \texttt{line\_buffer.v}: Bộ đệm 3 dòng để tạo cửa sổ 3×3 cho convolution
    \item \texttt{sobel\_kernel.v}: Tính toán gradient Gx, Gy theo kernel Sobel
    \item \texttt{edge\_mag.v}: Tính độ lớn biên (Manhattan distance) và bão hòa đầu ra
\end{itemize}

\section{Công Việc Đã Thực Hiện}

\subsection{Phát Triển Testbench Video (\texttt{tb\_sobel\_video.v})}
Testbench mới được thiết kế để:
\begin{itemize}
    \item Đọc metadata từ \texttt{video\_meta.txt} (số khung hình, độ phân giải, FPS)
    \item Phát lại từng pixel theo đúng timing vsync/href của chuẩn video
    \item Theo dõi tọa độ pixel qua pipeline 4 tầng để gắn nhãn chính xác
    \item Chỉ ghi các pixel nằm trong vùng hợp lệ ROI: hàng $\geq 2$ và cột $\geq 1$ (do giới hạn cửa sổ 3×3)
    \item Kiểm tra số lượng mẫu đầu ra khớp với kỳ vọng $(H-2) \times (W-1)$ mỗi khung
\end{itemize}

\textbf{Điểm cải tiến:} Pipeline tracking đảm bảo không ghi nhầm pixel nằm ngoài vùng hợp lệ, tránh lỗi kích thước file đầu ra.

\subsection{Script Chuẩn Bị Dữ Liệu (\texttt{prep\_video\_rgb565.py})}
Script Python thực hiện:
\begin{itemize}
    \item Đọc video từ file hoặc tự sinh clip tổng hợp (gradient màu) nếu không có đầu vào
    \item Resize mỗi frame về $640 \times 480$ pixel
    \item Chuyển đổi BGR → RGB → RGB565 (5-6-5 bit packing)
    \item Ghi stream nhị phân theo định dạng little-endian để khớp với \texttt{\$fgetc} trong Verilog
    \item Xuất metadata text file cho testbench
\end{itemize}

\textbf{Hỗ trợ nhiều nguồn:} Có thể đọc từ file MP4, thư mục ảnh PNG, hoặc tự sinh video fallback.

\subsection{Mô Hình Tham Chiếu Phần Mềm (\texttt{compare\_sobel\_output.py})}
Script so sánh bao gồm:
\begin{itemize}
    \item Giải nén RGB565 → RGB888 theo đúng quy tắc bit expansion của RTL
    \item Tính grayscale với công thức trọng số giống RTL: $(77R + 151G + 28B) >> 8$
    \item Áp dụng convolution Sobel với xử lý biên giống RTL (wrap-around cho cột cuối)
    \item Tính các chỉ số: số pixel khác biệt, sai số tuyệt đối max/mean, PSNR
    \item Xuất báo cáo JSON và video diff (hiển thị 3 cột: input, expected, actual)
\end{itemize}

\textbf{Độ chính xác:} Mô hình phần mềm mirror chính xác từng bước tính toán của RTL để phát hiện sai lệch nhỏ nhất.

\subsection{Tạo Video Chuyển Động Tổng Hợp (\texttt{generate\_motion\_video.py})}
Script sinh video test với các đối tượng di chuyển:
\begin{itemize}
    \item Thanh ngang dao động theo sin
    \item Hình vuông di chuyển chéo
    \item Hình tròn pulse (thay đổi bán kính)
    \item Background gradient để tạo cạnh mềm
\end{itemize}

\textbf{Ý nghĩa:} Tạo kích thích có nhiều loại biên (mạnh, yếu, chuyển động) để kiểm tra toàn diện.

\subsection{Tích Hợp Makefile}
Thêm target \texttt{video} vào \texttt{sim/Makefile}:
\begin{lstlisting}[language=make]
video:
    python ../scripts/prep_video_rgb565.py --output-dir ../data
    iverilog -g2012 -Wall -s tb_sobel_video -o sim_video.vvp \
        [danh sách module RTL] tb_sobel_video.v
    vvp sim_video.vvp
    python ../scripts/compare_sobel_output.py \
        --input ../data/video_in.rgb \
        --output ../data/video_out.rgb \
        --meta ../data/video_meta.txt \
        --report ../data/video_report.json \
        --diff-video ../data/video_compare.mp4
\end{lstlisting}

\textbf{Target clean} cũng được cập nhật để xóa tất cả artifact video.

\section{Kết Quả Kiểm Chứng}

\subsection{Lệnh Thực Thi}
\begin{lstlisting}[language=bash]
cd sim
make video
\end{lstlisting}

\subsection{Dữ Liệu Đầu Vào}
\begin{itemize}
    \item Video: \texttt{data/moving\_object.mp4} (120 frames, 30 fps)
    \item Độ phân giải: $640 \times 480$ pixels
    \item Định dạng: RGB565 packed little-endian
    \item Số frames mô phỏng: 30 (có thể tùy chỉnh bằng \texttt{--max-frames})
\end{itemize}

\subsection{Chỉ Số So Sánh}
Kết quả từ \texttt{data/video\_report.json}:

\begin{table}[h]
    \centering
    \begin{tabular}{@{}ll@{}}
        \toprule
        \textbf{Chỉ số} & \textbf{Giá trị} \\
        \midrule
        Số frame xử lý & 30 \\
        Kích thước frame & $640\times480$ (vùng hợp lệ $639\times478$) \\
        Tổng số pixel so sánh & 9\,163\,260 \\
        Số pixel sai khác & 138\,407 (1.51\%) \\
        Sai số tuyệt đối lớn nhất & 78 mức xám \\
        Sai số tuyệt đối trung bình & 0.9077 mức xám \\
        PSNR & 35.71 dB \\
        \bottomrule
    \end{tabular}
    \caption{Kết quả so sánh RTL vs phần mềm cho lần chạy \texttt{make video} gần nhất.}
    \label{tab:metrics}
\end{table}

\subsection{Đánh Giá}
\begin{itemize}
    \item \textbf{Tích cực:} Số lượng pixel đầu ra khớp 100\% với kỳ vọng $(H-2) \times (W-1) \times \text{frames}$, chứng tỏ testbench đã gating đúng vùng ROI.
    \item \textbf{Sai lệch nhỏ:} Tỷ lệ mismatch 1.51\% với PSNR 35.71 dB cho thấy kết quả tương đối tốt. Sai số chủ yếu tập trung ở biên frame (có thể do khác biệt trong xử lý wrap-around hoặc hằng số scaling).
    \item \textbf{Video diff:} File \texttt{data/video\_compare.mp4} cho thấy sự khác biệt chủ yếu ở các pixel biên mạnh, phù hợp với giả thuyết về vấn đề scaling/saturation.
\end{itemize}

\section{Các Artifact Tạo Ra}

\begin{table}[h]
    \centering
    \begin{tabular}{@{}ll@{}}
        \toprule
        \textbf{File} & \textbf{Mô tả} \\
        \midrule
        \texttt{data/video\_in.rgb} & Stream RGB565 đầu vào cho testbench \\
        \texttt{data/video\_meta.txt} & Metadata (frames, width, height, fps) \\
        \texttt{data/video\_out.rgb} & Kết quả Sobel từ mô phỏng RTL \\
        \texttt{data/video\_report.json} & Báo cáo so sánh định lượng \\
        \texttt{data/video\_compare.mp4} & Video trực quan hóa (3 cột: input/expected/actual) \\
        \bottomrule
    \end{tabular}
    \caption{Danh sách artifact được sinh ra bởi luồng kiểm chứng.}
\end{table}

\section{Nhận Xét và Hướng Phát Triển}

\subsection{Những Gì Đã Đạt Được}
\begin{itemize}
    \item ✓ Xây dựng thành công luồng kiểm chứng video tự động end-to-end
    \item ✓ Testbench Verilog mô phỏng chính xác timing video thực (vsync/href)
    \item ✓ Mô hình phần mềm tham chiếu mirror RTL ở mức bit-accurate
    \item ✓ Tích hợp hoàn chỉnh vào Makefile với 1 lệnh duy nhất (\texttt{make video})
    \item ✓ Sinh được video diff trực quan để review nhanh
    \item ✓ Độ phân giải 640×480 và hỗ trợ mở rộng cho video bất kỳ
\end{itemize}

\subsection{Vấn Đề Cần Khắc Phục}
\begin{itemize}
    \item \textbf{Sai lệch số học nhỏ (1.51\%):} Cần đối chiếu lại công thức magnitude trong RTL (\texttt{edge\_mag.v}) với phần mềm, đặc biệt là thứ tự shift và saturation.
    \item \textbf{Border handling:} Kiểm tra lại logic wrap-around ở cột $W-1$ trong \texttt{line\_buffer.v} có khớp với numpy \texttt{np.roll} hay không.
    \item \textbf{Tối ưu tốc độ mô phỏng:} Với video dài, có thể thêm \texttt{+define+FAST\_SIM} để tắt debug prints.
\end{itemize}

\subsection{Bước Tiếp Theo}
\begin{enumerate}
    \item Debug chi tiết các pixel sai khác bằng cách in ra tọa độ và giá trị gradient (Gx, Gy) trong testbench.
    \item So sánh từng bước tính toán (grayscale, window, gradient, magnitude) giữa RTL và Python cho 1 frame.
    \item Sau khi đạt PASS (0 mismatches), chạy regression với nhiều video khác nhau (real-world clips).
    \item Tích hợp vào CI pipeline: fail build nếu PSNR < ngưỡng hoặc mismatch > 0\%.
\end{enumerate}

\section{Kết Luận}

Đồ án đã hoàn thành mục tiêu xây dựng hệ thống kiểm chứng video cho bộ lọc Sobel RTL. Luồng tự động từ video đầu vào → mô phỏng → so sánh → báo cáo đã hoạt động ổn định. Kết quả mô phỏng cho thấy RTL xử lý đúng timing và số lượng pixel, với độ chính xác cao (PSNR > 35 dB). 

Các artifact sinh ra (JSON report, diff video) cung cấp đầy đủ thông tin để đánh giá chất lượng thiết kế và debug nhanh chóng. Hệ thống sẵn sàng được mở rộng cho các kịch bản test phức tạp hơn và tích hợp vào quy trình phát triển chuyên nghiệp.

\vspace{1cm}

\noindent\textbf{Người thực hiện:} Nhóm Sobel Project \\
\textbf{Ngày hoàn thành:} \today

\end{document}
